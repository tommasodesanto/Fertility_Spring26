\documentclass[11pt]{article}
\usepackage[fleqn]{amsmath}
\usepackage{amssymb}
\usepackage{amsthm}
\usepackage{geometry}
\usepackage{enumitem}
\usepackage{hyperref}
\usepackage{natbib}

\geometry{margin=1in}
\setlength{\parskip}{0.6em}
\setlength{\parindent}{0pt}

% Theorem environments
\newtheorem{theorem}{Theorem}
\newtheorem{proposition}{Proposition}
\newtheorem{lemma}{Lemma}
\newtheorem{corollary}{Corollary}
\newtheorem{definition}{Definition}
\newtheorem{assumption}{Assumption}
\newtheorem{remark}{Remark}

\title{Fertility, Housing, and Location Decisions}
\author{Tommaso De Santo}
\date{\today}

\begin{document}
\maketitle

\begin{abstract}
\end{abstract}
\newpage

\section{Introduction}

\section{Model}

\textbf{Notation for locations.} The set of locations is $\mathcal{I}=\{1,\dots,I\}$; generic locations are indexed by $i,j\in\mathcal{I}$. In any state, the current (origin) location is denoted $i\in\mathcal{I}$ and a candidate destination by $i'\in\mathcal{I}$. Location amenities are $\mathcal{E}_{i'}>0$. Location taste shocks are $\varepsilon^{\,i'}$ with i.i.d.\ Type-I extreme value (Gumbel) distribution. Location choice shares are $\pi^{\,i'}(\cdot)$ and location-stage values are $V^{\,i'}(\cdot)$. The (inverse) scale parameter governing dispersion in location choice is denoted $\nu_{\ell}>0$.

The economy features continuous age $a\in[0,A]$ and calendar time $t$. Shock ages are discrete $\mathcal{A}^s=\{0,1,\dots,A-1\}$. Working ages are $a\in[0,A_R]$; fertility is feasible on $a\in[0,A_f]$; children mature at $A_m$.

\subsection{Preferences}

Let housing services be $\mathbf{h}\equiv \chi\,h + h^R$, where $h$ is owner-occupied stock (adjusted only at shock ages) and $h^R$ is the rental flow (adjustable between shocks). Flow utility is Stone--Geary CRRA in $(c,\mathbf{h})$ with a \emph{single} curvature:
\[
u(c,\mathbf{h};n)
=\frac{\big(c-\bar c(n)\big)^{\,1-\sigma}}{1-\sigma}
+\kappa_h(n)\,\frac{\big(\mathbf{h}-\bar h(n)\big)^{\,1-\sigma}}{1-\sigma},
\qquad \sigma>0,\ \ \kappa_h'(n)\ge 0,
\]
where $\bar c(n)$ and $\bar h(n)$ are subsistence levels that may depend on parity $n$. In applications we allow a fixed ``first-child'' housing jump and per-child housing needs:
\[
\bar c(n)=\bar c_0+\bar c_1\,n,\qquad
\bar h(n)=\bar h_0+\bar h_{\mathrm{jump}}\cdot \mathbb{I}\{n\ge 1\}+\bar h_1\,n.
\]
We impose supernumerary lower bounds $c-\bar c(n)\ge c_0>0$ and $\mathbf{h}-\bar h(n)\ge h_0>0$, as well as $h^R\ge 0$.

\subsection{Earnings}
Households supply labor inelastically when $a\le A_R$ and earn $w_{it}(a,t)$ in location $i$ (specified in the production block below).

\subsection*{Mobility and location choice}

At shock ages, households draw an idiosyncratic taste shock for each destination $i'$, $\varepsilon^{\,i'}$, i.i.d.\ Type-I extreme value. Amenities $\mathcal{E}_{i'}$ and multiplicative moving wedges $\mu_{i i'}\in(0,\infty)$ scale the continuation value in destination $i'$.

Given state $(b,h,i,a,n,a_n,t)$ prior to relocation, moving to $i'$ resets liquid wealth and owner stock to
\begin{align*}
  \tilde b_t^{\,i'}(b,h,i) &= b + \mathbb{I}(i'\neq i)\,(1-\psi)\,p_{it}\,h,\\
  \tilde h_t^{\,i'}(h,i)   &= \mathbb{I}(i'=i)\,h,
\end{align*}
i.e., if the household moves it sells the origin house at net proceeds $(1-\psi)p_{it}h$ and arrives as a renter; if it stays it carries the owned stock.

\subsection*{Housing}

Rental sizes $h^R$ lie in $\mathcal{R}\subset\mathbb{R}_+$; owner sizes $h$ lie in $\mathcal{H}\subset\mathbb{R}_+$. Housing services are $\mathbf{h}(h,h^R)=\chi h+h^R$. Owner housing depreciates at rate $\delta$; owners pay a property tax $\tau_H$ on the house value; buying/selling incurs a transaction wedge $\psi\in[0,1)$.

\subsection{Fertility}

At shock ages, agents receive fertility shocks $\varepsilon^n$ (i.i.d.\ Type-I extreme value) and choose parity $n\in\{0,1,2,3\}$. Having children raises housing taste via $\kappa_h(n)$ and changes subsistence needs via $\bar c(n)$ and $\bar h(n)$. For now parity is chosen at a shock and children age deterministically to $A_m$ (we can later allow sequential fertility). The (inverse) scale parameter governing dispersion in fertility choice is denoted $\nu^n>0$.

\subsection{Entrants and local bequest recycling}

Newborns (age $a=0$) enter as renters with $h_0=0$. Their liquid wealth at entry is financed by local bequests. If deaths in $i$ carry $(b,h)$, the bequeathable estate is
\[
\mathrm{estate}_i(b,h)=b+(1-\psi)p_{it}h.
\]
Let $E_i$ denote the aggregate \emph{estate} flow in $i$ (notation distinct from amenity $\mathcal{E}_i$) and $E_i^0$ the newborn flow in $i$. A fraction $\varphi\in[0,1]$ is redistributed equally to newborns:
\[
T_i=\frac{\varphi\,E_i}{\max\{E_i^0,\varepsilon\}},\qquad b_0(i)=b_{\mathrm{entry}}+T_i,
\]
with $\varepsilon>0$ small. Budget balance: $\sum_i T_iE_i^0=\varphi\sum_i E_i$. (Utility bequests are separate from financial estates.)

\subsection{Portfolio and savings}

Liquid wealth $b$ earns $q$. Between shock ages, owners cannot rent ($h^R=0$) and renters can adjust $h^R\ge 0$. A collateral constraint must hold after tenure/size choices:
\[
b\ge -\phi\,p_{it}h.
\]
Between shocks, liquid wealth evolves as
\[
\dot{b}= w_{it}(a,t)+ q b -c - r_{it}h^R - (\delta+\tau_H) p_{it}h.
\]
At shock ages, $b$ adjusts discretely due to relocation, tenure/size moves, and bequests to children at $A_m$:
\[
\tilde b_t^{\,\text{beq}}(b,h,i,n,b^{\mathrm{beq}})=b-n\,b^{\mathrm{beq}}.
\]

\subsection{Household problem and value functions}

Let the household state be $\Omega=(b,h,i,a,n,a_n,t)$. Between shocks, the HJB is
\begin{align}
\rho V_t(\Omega)
= \max_{c-\bar c(n)\ge c_0,\;h^R\ge 0}\Big\{
u\!\big(c,\chi h+h^R;n\big)
+ V_b\,\dot b
+ V_a
\Big\},
\quad
\dot b=w_{it}(a,t)+qb-c-r_{it}h^R-(\delta+\tau_H)p_{it}h,
\end{align}
with Kuhn--Tucker conditions for $h^R\ge 0$ and $b\ge -\phi p_{it}h$.

At shock ages, the sequence is \emph{fertility $\to$ location $\to$ tenure/size}. We define the three blocks.

\paragraph{Tenure/size stage.}
Given a chosen destination $i'$, define the value from optimally adjusting $h'$:
\begin{align}
  V_t^{H}(b,h,i',a,n,a_n)
  \;=\; \max_{h'\in \mathcal{H}_t(b,h,i')}\; V_t\!\big(\tilde b_t^{\,H}(b,h,h',i'),\,h',\,i',\,a,\,n,\,a_n,\,t\big),
\end{align}
where $\tilde b_t^{\,H}(b,h,h',i')=b+\mathbb{I}(h'\neq h)\,\big[(1-\psi)p_{i't}h-p_{i't}h'\big]$ and
$\mathcal{H}_t(b,h,i')=\{h'\in\mathcal{H}:\tilde b_t^{\,H}\ge -\phi p_{i't}h'\}$.

\paragraph{Location stage.}
Define the (systematic) value of choosing destination $i'$ as
\[
V_t^{\,i'}(b,h,i,a,n,a_n)
\;=\;
V_t^{H}\!\big(\tilde b_t^{\,i'}(b,h,i),\,\tilde h_t^{\,i'}(h,i),\,i',\,a,\,n,\,a_n\big).
\]
With i.i.d.\ Type-I extreme value shocks, we model the \emph{choice index} as
\[
U_t^{\,i'}(b,h,i,a,n,a_n)
=\nu_{\ell}\,V_t^{\,i'}(b,h,i,a,n,a_n)+\log\!\big(\mathcal{E}_{i'}\,\mu_{i i'}\big)+\varepsilon^{\,i'}.
\]
This yields the logit share and the inclusive value (up to an additive constant):
\begin{align}
\pi_t^{\,i'}(b,h,i,a,n,a_n)
&=\frac{\big(\mathcal{E}_{i'}\,\mu_{i i'}\big)\exp\!\big(\nu_{\ell}V_t^{\,i'}(b,h,i,a,n,a_n)\big)}
{\sum_{j'\in\mathcal{I}}\big(\mathcal{E}_{j'}\,\mu_{i j'}\big)\exp\!\big(\nu_{\ell}V_t^{\,j'}(b,h,i,a,n,a_n)\big)},\\
V_t^{I}(b,h,i,a,n,a_n)
&=\frac{1}{\nu_{\ell}}\log\!\left(\sum_{j'\in\mathcal{I}}\big(\mathcal{E}_{j'}\,\mu_{i j'}\big)\exp\!\big(\nu_{\ell}V_t^{\,j'}(b,h,i,a,n,a_n)\big)\right).
\end{align}
In computation on a BGP, the logit kernel is applied to detrended continuation values (equivalently, to levels after dividing by the common trend), which preserves stationarity.

\paragraph{Fertility stage.}
Given i.i.d.\ Type-I extreme value shocks for parity choices and continuation values $V_t^{I}$,
\begin{align}
\pi_t^{\,n'}(b,h,i,a,a_n)
&=
\frac{\exp\!\big(\nu^n\,V_t^{I}(b,h,i,a,n',a_n)\big)}
{\sum_{m\in\{0,1,2,3\}}\exp\!\big(\nu^n\,V_t^{I}(b,h,i,a,m,a_n)\big)},
\\
V_t^{n}(b,h,i,a,a_n)
&=
\frac{1}{\nu^n}\log\!\left(\sum_{m\in\{0,1,2,3\}}
\exp\!\big(\nu^n\,V_t^{I}(b,h,i,a,m,a_n)\big)\right).
\end{align}
We adopt $V_t^{n}$ as the post-shock value $V_t$ at shock ages.

\subsection{Demography (Euler--Lotka)}

Let survival $\ell(a)$ and age-specific fertility $m(a)$; let $n$ denote the population growth rate. The Euler--Lotka condition is
\[
1=\int_0^A e^{-n a}\,\ell(a)\,m(a)\,da
\quad
\Big(\text{or } 1=\sum_{a\in\mathcal{A}^s} e^{-n a}\,\ell(a)\,m(a)\Delta a\text{ in discrete shock ages}\Big).
\]
With deterministic death at $A$, the stationary age density is uniform iff $n=0$; for $n\neq 0$, $f(a)\propto e^{-n a}\,\ell(a)$.

\subsection{Production (tradeable good and \emph{construction in the main text})}

\paragraph{Tradeable good.} The numeraire is costlessly traded. Competitive firms in $i$ produce
\[
Y_{it}=A_i\,Z_i(L_{it})\,L_{it},\qquad Z_i(L_{it})=L_{it}^{\alpha_i},
\]
yielding wage $w_{it}=A_i\,L_{it}^{\alpha_i}$.

\paragraph{Residential construction (irreversibility).} Developers transform the numeraire into new floorspace $\Upsilon_{it}$ with technology
\[
\Upsilon_{it}=Z^h_{it}\,K_{it},
\]
so the unit cost of new floorspace is $1/Z^h_{it}$. Let $H_{it}$ be the city stock of floorspace; it evolves as
\[
\dot H_{it}=\Upsilon_{it}-\delta H_{it},\qquad \Upsilon_{it}\ge 0 \quad\text{(irreversibility)}.
\]
Perfect competition implies the complementarity (Kuhn--Tucker) system:
\begin{align}
p_{it} \;\ge\; \frac{1}{Z^h_{it}},\qquad
\Upsilon_{it}\;\ge\;0,\qquad
\bigg(p_{it}-\frac{1}{Z^h_{it}}\bigg)\Upsilon_{it}=0.
\end{align}
\emph{Housing market clearing} equates the \emph{flow} of net new demand to construction net of depreciation:
\begin{align}
\dot H_{it}
&= \Upsilon_{it}-\delta H_{it}
= \frac{d}{dt}\,\Big[\underbrace{H^d_{it}}_{\text{occupied floorspace}}\Big]-\delta H_{it},
\\
H^d_{it}
&=\int\!\!\!\int\!\!\!\int \big(h^R(\Omega)+h\big)\; g_t(\Omega)\; db\,dh\,da,
\end{align}
with $g_t$ the cross-sectional density. In steady growth, these conditions pin down $p_{it}$ jointly with $H_{it}$ and $\Upsilon_{it}$. (A computational simplification that replaces the complementarity with an isoelastic static supply is in Appendix~\ref{app:supply_simplification}.)

\section{Equilibrium and Balanced Growth Path}

\subsection{Equilibrium}

Fix fundamentals $\{\sigma,\kappa_h(n),\chi,q,\delta,\tau_H,\phi,\psi,\bar c(\cdot),\bar h(\cdot),\{A_i\},\{Z^h_{it}\}\}$ and the i.i.d.\ extreme-value shock structure with dispersion parameters $\nu_{\ell}$ for locations and $\nu^n$ for fertility. An equilibrium is a collection
\begin{align*}
\Big\{ V_t(\Omega),\; \text{policies }(c,h^R)\text{ between shocks},\;
\text{discrete choice kernels }(\pi^{\,n},\pi^{\,i'},\pi^{\,H}) \text{ at shocks},\;
g_t(\Omega),\; \{p_{it},r_{it},w_{it},H_{it},\Upsilon_{it}\}_{i\in\mathcal I}\Big\}
\end{align*}
such that:
\begin{enumerate}[label=(E\arabic*)]
\item \textbf{Household optimality.} Between shocks $V_t$ solves the HJB with borrowing and nonnegativity constraints; at shocks, fertility $\to$ location $\to$ tenure/size decisions are optimal and generate choice probabilities $\pi^{\,n},\pi^{\,i'},\pi^{\,H}$ consistent with extreme-value aggregation.
\item \textbf{Cross-sectional consistency.} $g_t$ solves the KF between shocks and is updated by the conservative reallocation induced by $(\pi^{\,n},\pi^{\,i'},\pi^{\,H})$ at shock ages.
\item \textbf{User cost (levels).} Rents and prices satisfy
\[
r_{it}=\big(q+\delta+\tau_H-g_{p,it}\big)\,p_{it},
\qquad
g_{p,it}\equiv \frac{d}{dt}\log p_{it}.
\]
On a BGP with $p_{it}=e^{\gamma t}\hat p_i$, this reduces to $\hat r_i=(q+\delta+\tau_H-\gamma)\,\hat p_i$.
\item \textbf{Construction and market clearing.} For each $i,t$,
\[
p_{it}\ge \frac{1}{Z^h_{it}},\quad \Upsilon_{it}\ge 0,\quad
\bigg(p_{it}-\frac{1}{Z^h_{it}}\bigg)\Upsilon_{it}=0,\quad
\dot H_{it}=\Upsilon_{it}-\delta H_{it}=\frac{d}{dt}H^d_{it}-\delta H_{it}.
\]
\end{enumerate}

\subsection{Balanced Growth Path (BGP) and hats}

Let hats denote detrended variables, $\hat z=e^{-\gamma t}z$ for $z\in\{b,c,h,h^R,p_i,r_i,w_i,H_i,\Upsilon_i\}$. We take the value scaling
\[
V_t(\Omega)=e^{(1-\sigma)\gamma t}\,v(\hat b,\hat h,\hat\Omega),\qquad \hat\Omega=(i,a,n,a_n).
\]

\paragraph{HJB in hats.} Between shocks,
\begin{align}
(\rho-(1-\sigma)\gamma)\,v
=\max_{\hat c-\bar c(n)\ge c_0,\;\hat h^{\,R}\ge 0}\Big\{
u(\hat c,\chi\hat h+\hat h^{\,R};n)
+ v_{\hat b}\,\dot{\hat b}
+ v_a
- \gamma\,\hat h\,v_{\hat h}
\Big\},
\quad
\dot{\hat b}=\hat w_i+(q-\gamma)\hat b-\hat c-\hat r_i \hat h^{\,R}-(\delta+\tau_H)\hat p_i\hat h.
\end{align}

\paragraph{KF in hats (share normalization).} The stationary KF between shocks is
\[
\partial_a\phi+\partial_{\hat b}(\phi\,\dot{\hat b})+\partial_{\hat h}(\phi\,(-\gamma\hat h))=0,
\]
with discrete, conservative updates at shock ages using $(\pi^{\,n},\pi^{\,i'},\pi^{\,H})$ in that order.
We adopt the share normalization
\[
g_t(\Omega)=e^{-2\gamma t}\,\phi(\hat b,\hat h,\hat\Omega),
\]
so that $\phi$ integrates to one by construction; totals are recovered by multiplying by population when aggregating.

\paragraph{User cost in hats.} $\hat r_i=(q+\delta+\tau_H-\gamma)\hat p_i$.

\paragraph{Construction in hats.} From levels,
\[
\dot H_{it}=\Upsilon_{it}-\delta H_{it}.
\]
With $\hat H_i=e^{-\gamma t}H_{it}$ and $\hat\Upsilon_i=e^{-\gamma t}\Upsilon_{it}$,
\[
\dot{\hat H}_i=0\quad\Longleftrightarrow\quad \hat\Upsilon_i=(\delta-\gamma)\,\hat H_i
\]
on a stationary BGP. Combined with the level complementarity $p_{it}\ge 1/Z^h_{it}$, $\Upsilon_{it}\ge 0$, $\big(p_{it}-1/Z^h_{it}\big)\Upsilon_{it}=0$, this pins down prices jointly with stocks at the BGP.

\paragraph{Demand in hats.} $H_i^d=\int(\hat h^R+\hat h)\,\phi\,d\hat b\,d\hat h\,da$ and market clearing requires $\frac{d}{dt}\hat H_i^d=0$ on a strict BGP, which is consistent with $\hat\Upsilon_i=(\delta-\gamma)\hat H_i$.

\subsection*{First-order conditions (interior case, hats)}

With Stone--Geary marginal utilities
\[
u_c(\hat c,\mathbf{\hat h};n)=\big(\hat c-\bar c(n)\big)^{-\sigma},
\qquad
u_{\mathbf{h}}(\hat c,\mathbf{\hat h};n)=\kappa_h(n)\big(\mathbf{\hat h}-\bar h(n)\big)^{-\sigma},
\quad \mathbf{\hat h}\equiv \chi\hat h+\hat h^{\,R},
\]
the interior between-shock policies in hats satisfy
\[
u_c(\hat c,\chi\hat h+\hat h^{\,R};n)=v_{\hat b},\qquad
\frac{u_{\mathbf{h}}(\hat c,\chi\hat h+\hat h^{\,R};n)}{u_c(\hat c,\chi\hat h+\hat h^{\,R};n)}=\hat r_i.
\]
Equivalently,
\[
\hat c=\bar c(n)+v_{\hat b}^{-1/\sigma},\qquad
\chi\hat h+\hat h^{\,R}=\bar h(n)+\Big(\tfrac{\kappa_h(n)}{\hat r_i v_{\hat b}}\Big)^{\!1/\sigma},\qquad
\hat h^{\,R}=\max\{0,\;(\cdot)-\chi\hat h\}.
\]
Kuhn--Tucker inequalities apply at the collateral wall and the rental corner.

\section{Computation}

\subsection{Supply simplification for numerics}\label{app:supply_simplification}

For computation, one may replace the construction complementarity by a static isoelastic \emph{approximation} around the BGP:
\[
H_i^s(\hat p_i)=\bar H_i\,\hat p_i^{\eta_i},\qquad \eta_i>0,
\]
and close the market with $H_i^d(\hat p_i)=H_i^s(\hat p_i)$. This delivers a stable price-only fixed point. The mapping to the construction block is determined by matching $(\hat p_i,\hat H_i)$ at the targeted BGP and interpreting $\eta_i$ as a local supply elasticity.

\section{Conclusion}
\pagebreak
\section{Appendix A: HJB and KFE in levels and hats (clean derivation)}
\label{app:cleanHJBKFE}

\paragraph{Price convention.} The \emph{tradable numeraire} has a common price $P^T_t\equiv 1$ in all locations. Symbols $p_{it},r_{it}$ denote the \emph{local (non-tradable) housing asset price and rent}, both quoted in the numeraire. Hence $p_{it}$ may differ by $i$; there is no conflict with the law of one price for tradables.

\subsection*{A.1 Levels HJB with explicit time dependence}
Let the between-shocks state be $\Omega=(b,h,i,a,n,a_n,t)$ with $\dot h=0$ (no owner adjustment between shocks), $\dot a=1$, and liquid wealth drift
\[
\dot b=w_{it}(a,t)+q b - c - r_{it} h^R - (\delta+\tau_H)p_{it} h .
\]
The correct HJB in levels (non-autonomous environment) is
\begin{equation}
\label{eq:HJB_levels}
\rho V(t,\Omega)
=\max_{c-\bar c(n)\ge c_0,\;h^R\ge 0}\Big\{
u\!\big(c,\chi h+h^R;n\big)
+\underbrace{\partial_t V}_{\text{explicit time}}
+ V_b\,\dot b
+ V_a
\Big\},
\end{equation}
with Kuhn–Tucker for $h^R\ge 0$ and the collateral constraint enforced at shocks.

\subsection*{A.2 Detrending and stationary HJB}
Let hats be $\hat z=e^{-\gamma t}z$ for $z\in\{b,c,h,h^R,w_i,p_i,r_i\}$ and scale values
\[
V(t,\Omega)=e^{(1-\sigma)\gamma t}\,v(\hat b,\hat h,i,a,n,a_n).
\]
Utility scales as
\[
u(e^{\gamma t}\hat c,\;e^{\gamma t}(\chi\hat h+\hat h^{\,R});n)
=e^{(1-\sigma)\gamma t}\,u(\hat c,\chi\hat h+\hat h^{\,R};n),
\]
since $\bar c(n)$ and $\bar h(n)$ are defined in detrended units.
Using $\partial_t \hat b=-\gamma \hat b$, $\partial_t \hat h=-\gamma \hat h$, and $\dot{\hat b}=(\dot b/e^{\gamma t})-\gamma \hat b$, the chain rule yields
\[
\partial_t V
=e^{(1-\sigma)\gamma t}\!\left[(1-\sigma)\gamma v -\gamma\hat b\,v_{\hat b}-\gamma\hat h\,v_{\hat h}\right].
\]
Moreover, $V_b=e^{-\sigma\gamma t}v_{\hat b}$ and $V_a=e^{(1-\sigma)\gamma t}v_a$. Substitute into \eqref{eq:HJB_levels} and divide by $e^{(1-\sigma)\gamma t}$ to obtain the stationary HJB in hats:
\begin{equation}
\label{eq:HJB_hats}
(\rho-(1-\sigma)\gamma)\,v
=\max_{\hat c-\bar c(n)\ge c_0,\;\hat h^{\,R}\ge 0}\Big\{
u(\hat c,\chi\hat h+\hat h^{\,R};n)
+ v_{\hat b}\,\dot{\hat b}
+ v_a
- \gamma\,\hat h\,v_{\hat h}
\Big\},
\end{equation}
with
\begin{equation}
\label{eq:bhat_drift}
\dot{\hat b}
=\hat w_i+(q-\gamma)\hat b-\hat c-\hat r_i \hat h^{\,R}-(\delta+\tau_H)\hat p_i\hat h.
\end{equation}
This matches the interior FOCs:
\[
(\hat c-\bar c(n))^{-\sigma}=v_{\hat b},\qquad
\frac{\kappa_h(n)(\chi\hat h+\hat h^{\,R}-\bar h(n))^{-\sigma}}{(\hat c-\bar c(n))^{-\sigma}}=\hat r_i,
\]
i.e.
\[
\hat c=\bar c(n)+v_{\hat b}^{-1/\sigma},\ 
\chi\hat h+\hat h^{\,R}=\bar h(n)+\big(\kappa_h(n)/(\hat r_i v_{\hat b})\big)^{1/\sigma}.
\]

\subsection*{A.3 KFE in levels and hats}
Let $g_t(b,h,i,a,n,a_n)$ be the (mass or share) density between shocks. With $\dot h=0$ and $\dot a=1$,
\begin{equation}
\label{eq:KFE_levels}
\partial_t g_t
+ \partial_b\!\big(g_t\,\dot b\big)
+ \partial_a g_t
= 0,
\qquad \dot b \text{ as above.}
\end{equation}

\paragraph{Share normalization (recommended).} Define
\[
g_t(\Omega) = e^{-2\gamma t}\,\phi(\hat b,\hat h,i,a,n,a_n),
\qquad \int\phi=1,
\]
so the Jacobian of $(b,h)\mapsto(\hat b,\hat h)$ contributes exactly $e^{-2\gamma t}$. Then \eqref{eq:KFE_levels} is equivalent to the stationary PDE
\begin{equation}
\label{eq:KFE_hats_share}
\partial_a\phi
+\partial_{\hat b}\big(\phi\,\dot{\hat b}\big)
+\partial_{\hat h}\big(\phi\,(-\gamma\hat h)\big)=0,
\end{equation}
with $\dot{\hat b}$ from \eqref{eq:bhat_drift}. At shock ages, $\phi$ undergoes a \emph{conservative} reallocation via the jump operator $\mathcal{J}$ implementing the sequence fertility $\to$ location $i\to i'$ $\to$ tenure/size.

\paragraph{Mass normalization (alternative).} If one prefers to keep \emph{levels} in the density,
\[
g_t(\Omega)=e^{-(2\gamma+n_t)t}\,\phi(\hat b,\hat h,i,a,n,a_n),
\]
with $n_t\equiv \dot N_t/N_t$ the endogenous population growth rate defined in Appendix~\ref{app:flows_equilibrium}. The PDE for $\phi$ is still \eqref{eq:KFE_hats_share}; the only difference is whether population growth sits inside $g_t$ or outside in a separate $N_t$ equation.

\paragraph{Boundary and constraints.} No-flux at the collateral wall and at $h^R=0$ corners; age boundary conditions at $a=0$ and $a=A$ are imposed by the flow equations for entrants and exits (Appendix~\ref{app:flows_equilibrium}); house size $h$ is constant between shocks (hence only the $-\gamma\hat h$ drift in hats).

\bigskip
\noindent\textit{Verification.} The cancellation of $-\gamma\hat b v_{\hat b}$ between $\partial_t V$ and $V_b\dot b$ terms is exact; the residual $-\gamma \hat h v_{\hat h}$ arises because $\dot h=0$ between shocks. Equation \eqref{eq:KFE_hats_share} follows from the standard change-of-variables formula plus the drift identities above.

\section{Appendix B: Equilibrium via population flows (no Euler--Lotka)}
\label{app:flows_equilibrium}

\subsection*{B.1 Stocks, flows, and relocation}
Let $N_i(t)$ be the population in location $i$, and let $\phi_i(\hat b,\hat h,a,n,a_n)$ be the stationary share density (Appendix~\ref{app:cleanHJBKFE}). Define
\[
\mathcal{S}_i(t):=\{a\in[0,A]\text{ hitting a shock at }t\},\qquad
\omega_i(a,t):=\text{mass flow into age }a\text{ in }i \text{ per unit time}.
\]
Relocation flows at a shock age $a\in\mathcal{S}_i(t)$ are
\[
M_{i\to i'}(t;a)=N_i(t)\,\omega_i(a,t)\!\int\!\pi^{\,i'}(\hat b,\hat h,i,a,n,a_n)\,\phi_{i|a}(\hat b,\hat h,n,a_n)\,d\hat b\,d\hat h\,dn\,da_n,
\]
with $\phi_{i|a}$ the cross-section conditional on age $a$. Net relocation obeys $\sum_{i'}M_{i\to i'}=\sum_{j}M_{j\to i}$ by construction of the choice shares and jump operator.

\subsection*{B.2 Births and deaths}
Let $d(a)$ be the instantaneous death hazard (or deterministic exit at $A$). Let parity choice at age $a$ produce $n'\in\{0,1,2,3\}$ children. Then births in $i$ are
\[
B_i(t)=\sum_{a\in\mathcal{S}_i(t)\cap[0,A_f]}
N_i(t)\,\omega_i(a,t)\!\int\!\Big(\sum_{n'} n'\cdot \pi^{\,n'}(\hat b,\hat h,i,a,a_n)\Big)\,
\phi_{i|a}(\hat b,\hat h,n,a_n)\,d\hat b\,d\hat h\,dn\,da_n .
\]
Deaths are
\[
D_i(t)=\int_0^A d(a)\,N_i(t)\,f_i(a,t)\,da,
\]
with $f_i(a,t)$ the age density (the $a$-marginal of $\phi_i$).

\subsection*{B.3 Population laws of motion and growth}
Total population in location $i$ satisfies
\begin{equation}
\label{eq:Ni_flow}
\dot N_i(t)
=\underbrace{B_i(t)-D_i(t)}_{\text{natural growth}}
+\underbrace{\sum_{j\neq i}M_{j\to i}(t)-\sum_{j\neq i}M_{i\to j}(t)}_{\text{net relocation}}.
\end{equation}
Aggregate population $N(t)=\sum_i N_i(t)$ obeys
\begin{equation}
\label{eq:N_agg}
\dot N(t)=B(t)-D(t),\qquad B(t)=\sum_i B_i(t),\quad D(t)=\sum_i D_i(t),
\end{equation}
because relocations cancel in the aggregate. The endogenous growth rate is $n_t=\dot N(t)/N(t)$.

\subsection*{B.4 Wages, labor, and housing markets}
Working-age labor in $i$:
\[
L_i(t)=N_i(t)\int_0^{A_R} f_i(a,t)\,da \;\times\; (\text{participation }=1).
\]
Firms: $w_{it}=A_i\,L_i^{\alpha_i}$.  
Local housing demand (occupied floorspace) in hats:
\[
\hat H_i^d=\int\!\big(\hat h^{\,R}(\hat b,\hat h,i,a,n,a_n)+\hat h\big)\,\phi_i(\hat b,\hat h,a,n,a_n)\,d\hat b\,d\hat h\,da\,dn\,da_n .
\]
Level demand is $H_i^d(t)=N_i(t)\,\hat H_i^d\,e^{\gamma t}$.  
Construction: either (i) complementarity $p_{it}\ge 1/Z^h_{it}$, $\Upsilon_{it}\ge 0$, $(p_{it}-1/Z^h_{it})\Upsilon_{it}=0$, $\dot H_{it}=\Upsilon_{it}-\delta H_{it}$; or (ii) static isoelastic $H_i^s=\bar H_i\,p_{it}^{\eta_i}$.

\subsection*{B.5 Balanced Growth Path (BGP) without Euler--Lotka}
A BGP is a tuple $(\gamma,\{\hat w_i,\hat p_i,\hat r_i,\hat H_i,\hat \Upsilon_i\}_{i},\ v,\ \{\phi_i\}_i,\ \{N_i(t)\})$ s.t.:
\begin{enumerate}[label=(B\arabic*),leftmargin=1.35em]
\item \textbf{Stationary hats and policies.} HJB \eqref{eq:HJB_hats} holds; \(\{\hat w_i,\hat p_i,\hat r_i\}\) are constant over $t$; $\phi_i$ solves \eqref{eq:KFE_hats_share} with conservative shock updates.
\item \textbf{Common co-trend.} $w_{it}=e^{\gamma t}\hat w_i$, $p_{it}=e^{\gamma t}\hat p_i$, $r_{it}=e^{\gamma t}\hat r_i$, with user cost $\hat r_i=(q+\delta+\tau_H-\gamma)\hat p_i$.
\item \textbf{Flow balance.} There exists $n$ such that $N_i(t)=s_i\,N_0\,e^{nt}$ with constant shares $s_i\in[0,1]$ and \eqref{eq:Ni_flow} holds for each $i$. This determines $n$ \emph{from} $B_i,D_i,M_{i\to j}$; Euler–Lotka is not used.
\item \textbf{Markets.} For each $i$, either (A) complementarity pins $\hat p_i$ jointly with $\hat H_i$, $\hat\Upsilon_i=(\delta-\gamma)\hat H_i$; or (B) static supply clears $\hat H_i^d=\bar H_i\,\hat p_i^{\eta_i}$.
\item \textbf{Production consistency.} With $L_i(t)=s_i\,e^{nt}\int_0^{A_R}f_i(a)\,da$, wages co-trend at $g_{w,i}=\alpha_i n$. Stationary hats then require $\gamma=g_{w,i}$, hence either $\alpha_i$ common or $n$ chosen so that $\alpha_i n$ is common (we assume common $\alpha$ or equilibrating $A_i$ to ensure a single $\gamma$).
\end{enumerate}
Under (A), if construction is active in $i$, $p_{it}=1/Z^h_{it}$ implies $g_{p,i}=-g_{Z^h,i}$; stationary hats require $g_{Z^h,i}=-\gamma$ in all active cities. Under (B), $\hat p_i$ is set by static clearing; stationary hats require parameter combinations guaranteeing $g_p=\gamma$ (see Appendix~\ref{app:prices_BGP}).
  
\section{Appendix C: Tradable vs non-tradable prices and BGP conditions}
\label{app:prices_BGP}

\paragraph{Tradable numeraire.} The price of the tradable is equalized across locations: $P^T_{it}\equiv P^T_t\equiv 1$. All prices in the model are in this numeraire.

\paragraph{Local housing price.} $p_{it}$ is the price of a non-tradable local asset (residential floorspace). Differences in $p_{it}$ across $i$ are consistent with spatial frictions and local supply.

\begin{proposition}[Co-trending condition for a single-$\gamma$ BGP]
On a BGP with stationary hats, wages and local housing prices must co-trend at rate $\gamma$:
\[
g_w=\gamma,\qquad g_p=\gamma.
\]
Under construction complementarity, active cities require $g_{Z^h,i}=-\gamma$; under isoelastic supply $H_i^s=\bar H_i p_i^{\eta_i}$ and CRRA curvature $\sigma$, log-linear clearing implies
\[
g_p=\frac{n+\gamma}{\eta_i+1/\sigma}.
\]
Setting $g_p=\gamma$ gives the restriction $\eta_i=1-1/\sigma+1/\alpha$ (with common $\alpha$ so $\gamma=\alpha n$). If this restriction is violated, prices drift and a strict BGP does not exist (that case should be interpreted as transitional dynamics).
\end{proposition}

\paragraph{Implication.} There is no contradiction with the law of one price: $P^T$ (tradable) is common; $p_{it}$ (housing) is local. BGP existence requires common \emph{growth} of $p_{it}$ across cities equal to $\gamma$; levels may differ by $i$.
\section{Appendix D: Stationary Age Distribution on a BGP with \texorpdfstring{$n \neq 0$}{n ≠ 0}}
\label{app:age_distribution}

\subsection*{D.1 Demographic Accounting}

Let $N(t)$ denote aggregate population and $f(a,t)$ the age density such that $N(t) = \int_0^A f(a,t)\,da$. On a BGP with constant population growth rate $n$, we have $N(t) = N_0 e^{nt}$.

The age density evolves according to the McKendrick--von Foerster equation. With deterministic death at age $A$ (no mortality before $A$):
\begin{equation}
\frac{\partial f}{\partial t} + \frac{\partial f}{\partial a} = 0, \quad a \in [0,A).
\end{equation}

\subsection*{D.2 Stationary Age Structure}

\begin{proposition}[BGP Age Distribution]
On a BGP with population growth rate $n$, the age density takes the form
\[
f(a,t) = e^{nt} \cdot \psi(a), \qquad \psi(a) = \psi(0)\,e^{-na}.
\]
The normalized cross-sectional age distribution is
\begin{equation}
\label{eq:age_dist}
\tilde{f}(a) \equiv \frac{f(a,t)}{N(t)} = \frac{n}{1 - e^{-nA}}\,e^{-na}, \quad a \in [0,A].
\end{equation}
\end{proposition}

\begin{proof}
Substitute the ansatz $f(a,t) = e^{nt}\psi(a)$ into the PDE:
\[
n e^{nt}\psi(a) + e^{nt}\psi'(a) = 0 \implies \psi'(a) = -n\psi(a) \implies \psi(a) = \psi(0)e^{-na}.
\]
Normalization requires $\int_0^A \tilde{f}(a)\,da = 1$:
\[
\int_0^A C\,e^{-na}\,da = C \cdot \frac{1 - e^{-nA}}{n} = 1 \implies C = \frac{n}{1 - e^{-nA}}.
\]
\end{proof}

\begin{remark}[Limiting Cases]
As $n \to 0$, L'H\^opital's rule gives $\tilde{f}(a) \to 1/A$ (uniform). For $n > 0$ (growing population), young cohorts dominate; for $n < 0$ (shrinking population), old cohorts dominate:
\[
\frac{\tilde{f}(A)}{\tilde{f}(0)} = e^{-nA} 
\begin{cases}
< 1 & \text{if } n > 0,\\
= 1 & \text{if } n = 0,\\
> 1 & \text{if } n < 0.
\end{cases}
\]
\end{remark}

\subsection*{D.3 Implication for the Detrended KFE}

Under the share normalization $g_t(\Omega) = e^{-2\gamma t}\phi(\hat{b},\hat{h},\hat{\Omega})$, the stationary density $\phi$ must embed the $e^{-na}$ age structure. Decompose:
\[
\phi(\hat{b},\hat{h},i,a,n,a_n) = \tilde{f}(a) \cdot \phi_{|a}(\hat{b},\hat{h},i,n,a_n \mid a),
\]
where $\phi_{|a}$ is the conditional distribution given age $a$.

\paragraph{Implementation.} When propagating mass from age $a$ to $a + \Delta a$ in a discrete-time approximation, apply the demographic correction factor:
\begin{equation}
\label{eq:demographic_factor}
\phi(\cdot, a + \Delta a) = e^{-n\,\Delta a} \cdot \phi^{\text{post-transition}}(\cdot, a),
\end{equation}
where $\phi^{\text{post-transition}}$ is the density after within-period drift and discrete choice updates. This ensures the age marginal of $\phi$ matches \eqref{eq:age_dist}.

\paragraph{Normalization.} The total mass under the correct age weighting is
\[
\int \phi\,d\hat{b}\,d\hat{h}\,da = 1, \quad \text{where the } a\text{-marginal satisfies } \int_0^A \tilde{f}(a)\,da = 1.
\]
In discrete time with $J$ periods of length $\Delta a$:
\[
\sum_{j=1}^{J} \text{mass}(j) = 1, \quad \frac{\text{mass}(j)}{\sum_{j'}\text{mass}(j')} \approx \tilde{f}((j-1)\Delta a)\,\Delta a.
\]

\subsection*{D.4 Numerical Verification}

After solving the KFE, verify that the age marginals match theory:
\[
\text{mass}(j) \propto e^{-n(j-1)\Delta a}, \quad j = 1,\ldots,J.
\]
Define the diagnostic:
\[
\epsilon_{\text{age}} \equiv \max_{j} \left| \frac{\text{mass}(j)}{\sum_{j'}\text{mass}(j')} - \tilde{f}((j-1)\Delta a)\,\Delta a \right|.
\]
A well-implemented KFE should achieve $\epsilon_{\text{age}} < 0.01$.

\subsection*{D.5 Timing Convention at Shock Ages}

At shock ages $a \in \mathcal{A}^s$, households make discrete choices. The timing convention must be consistent between the HJB (backward) and KFE (forward).

\paragraph{Convention.} At the beginning of age $a$, the sequence of events is:
\begin{enumerate}[label=(\roman*)]
    \item \textbf{Fertility choice} (if $a \in [A_f^{\text{start}}, A_f^{\text{end}}]$ and $n = 0$): draw $\varepsilon^{n'}$ i.i.d.\ Type-I extreme value and choose parity $n' \in \{0,1,2,3\}$.
    \item \textbf{Location choice}: draw $\varepsilon^{i'}$ i.i.d.\ Type-I extreme value and choose destination $i' \in \mathcal{I}$.
    \item \textbf{Tenure/size choice}: choose $h' \in \mathcal{H}$.
    \item \textbf{Within-period flow}: consumption $c$, rental $h^R$, wealth accumulation $\dot{b}$.
    \item \textbf{Age advancement}: transition to $a + \Delta a$.
\end{enumerate}

\paragraph{HJB implementation.} Backward induction at age $j$:
\begin{align*}
    V^H(b, h, i', j, n, a_n) &= \max_{h' \in \mathcal{H}} V^{\text{flow}}(\tilde{b}^H, h', i', j, n, a_n), \\
    V^I(b, h, i, j, n, a_n) &= \frac{1}{\nu_{\ell}}\log\!\left( \sum_{i'\in\mathcal I} (\mathcal E_{i'}\mu_{ii'})\exp\big(\nu_{\ell}V^{i'}(b,h,i,j,n,a_n)\big) \right), \\
    V^{\text{fert}}(b, h, i, j, a_n) &= \frac{1}{\nu^n}\log\!\left( \sum_{n'=0}^{3} \exp\big(\nu^n V^I(b,h,i,j,n',a_n)\big) \right).
\end{align*}
For agents with $n = 0$ at fertile ages, the value function is $V(b, h, i, j, 0, a_n) = V^{\text{fert}}(b, h, i, j, a_n)$.

\paragraph{KFE implementation.} Forward iteration at age $j$:
\begin{enumerate}[label=(\roman*)]
    \item \textbf{Fertility}: For mass $\phi(b, h, i, j, 0, 1)$ (childless, no kids present), redistribute:
    \[
    \phi(b, h, i, j, n', a_n') \mathrel{+}= \pi^{n'}(b, h, i, j) \cdot \phi(b, h, i, j, 0, 1), \quad n' \in \{0,1,2,3\},
    \]
    where $a_n' = 0$ if $n' > 0$ (newborn children) and the child state updates accordingly.
    \item \textbf{Location}: For each $(n, a_n)$, redistribute mass across destinations using $\pi^{i'}(b, h, i, j, n, a_n)$.
    \item \textbf{Tenure}: Concentrate mass at optimal $h'$ given post-relocation wealth.
    \item \textbf{Drift}: Solve the KFE for within-period dynamics.
    \item \textbf{Advance}: $\phi(\cdot, j+1, \cdot) \mathrel{+}= \phi^{\text{post-drift}}(\cdot, j, \cdot)$.
\end{enumerate}

\begin{remark}[Timing consistency]
The fertility choice must occur \emph{before} within-period drift in the KFE, matching the HJB where fertility affects the value function at age $j$. If fertility is applied after drift (i.e., to mass at $j+1$), there is a one-period delay in child costs and housing demand, creating an inconsistency between the value function and the cross-sectional distribution.
\end{remark}

\end{document}
